\section{$WS_2$ CVD state of the art}

One of the major hurdles in development in the field of the TMDCs is the lack of scalable and reproducible method of producing large area samples of TMDCs. One of the most popular method of obtaining samples, especially for the purposes of further research, has been mechanical exfoliation from bulk crystals to single or few layer TMDCs. Said bulk crystals can be found naturally or synthesised using a chemical vapour transport (CVT) method. In this method the precursors are kept in a closed system together with carrier gas and the target substrate and the temperature gradient is applied across the system. The precursor powder (a mixture of metal and chalcogenide precursors) evaporates with the assist of the transport agent to then deposit on the target substrate over the course of days to weeks \cite{Reale2016}\cite{Schmidt2013}. Utilising this method it has been shown that single bulk crystals of $WS_2$, $MoS_2$, $WTe_2$, $MoSe_2$, $MoTe_2$ can be grown \cite{Reale2016}\cite{Schmidt2013}\cite{Al-Hilli1972}\cite{Brixner1962}\cite{Lenz1997}\cite{Brown1966}\cite{Sunil1997}\cite{Lenz1997}. Such grown crystals are then used to produced single and few layer TMDCs by micromechanical exfoliaiton and deposition onto a desired substrate. While as produced thin layers show good optical and electronic properties their size, thickness and distribution are not easily controllable.

Because of that a different method of direct synthesis of thin layers of TMDCs has been used. For the synthesis of $WS_2$ thin films several different W has been considered. The most common W precursor used is a $WO_3$ which tends to produce $WS_2$ samples with very low number of impurities. At the same time however it requires the high temperature (~1000 {\degree}C) to evaporate and sulfurise. As an alternative with much lower evaporation temperature (~300 {\degree}C) and high volatility the tungsten hexacarbonyl ($WC_6$) can be used. The downside of that precursor however is the potential contamination of the sample with carbon and oxygen. Finally the tungsten halides can form films of high purity, but at the same time their usage results in release of highly corrosive byproducts \cite{Reale2016}. As an alternative to regular CVD the metal organic chemical vapour deposition (MOCVD) has been used to produce uniform $WS_2$ films of ~65 nm thickness using $WC_6$ and $H_2S$ as precursors \cite{Chung1998}.

One of the methods of producing $WS_2$ thin films is a two step process where a thin film of W or $WO_3$ is first deposited onto the substrate and subsequently sulphurised. Those thin films of tungsten can be first deposited using electron beam evaporation or magnetron sputtering. In order to produce thin film of $WO_3$ the as produced W film can be then thermally oxidised. Finally the thin film of W or $WO_3$ is placed in a tubular furnace, like the one used in CVD, heated up to ~800 {\degree}C and exposed to chalcogen atmosphere. As grown $WS_2$ films of up to ~200nm in lateral size from W films retain the distribution and shape of the W films. The $WS_2$ films produced from $WO_3$ films showed triangular growths of up to micrometers in size . This suggests that in case of the sulphurisation of the thin layers of $WO_3$ the W and S species diffuse on the substrate nucleating $WS_2$ triangles and growing to coalesce with other islands resulting in distribution of $WS_2$ domains \cite{doi:10.1021/nn400971k}\cite{Reale2016}.