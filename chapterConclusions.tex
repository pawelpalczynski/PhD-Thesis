\chapter{Conclusions and future work} 

We have shown that the CVD synthesis of $WS_2$ using novel precursors can result in formation of high quality atomically thin flakes at low temperatures up to hundreds of $\mu m$ in size. In particular this method proves to result in better flakes in essentially every way in comparison to more common oxide precursor like $WO_3$. A very narrow PL of 36 meV and record high room temperature electron mobility of 52 $cm^2/VS$ proves quality that exceeds even a naturally occurring $WS_2$. The used precursors also avoids the problems of commonly used oxy-carbon based precursors of $W$ in that it does not contaminate the synthesis products and does not produce toxic by products. It has also been shown that Raman spectroscopy and PL spectroscopy can be used as a reliable tool to map the as-grown samples and easily extract properties of the material. The insights brought about through those characterisations can prove to be useful moving forward and applying them to a host of other TMDC materials and systems. Additionally the synthesis routes and the precursors developed here can prove useful for fabrication of high-quality large-scale low-cost flakes in future. Similarly the presented precursors and synthesis method can be expanded to other TMDCs such $MoS_2$, $WSe_2$ or $MoSe_2$ to produce material of quality higher than has been otherwise reported in literature.

The synthesis method developed for $WS_2$ has been then applied to produce $WSe_2$ flakes. By utilising the tungstic acid ($H_2WO_4$) and $Se$ powders a similar low temperature synthesis resulted in flakes of up to 30 $\mu m$ in size. Similarly the PL and Raman spectroscopy has been utilised to characterise the samples and a strong PL with FWHM of about 66 meV in monolayer $WSe_2$ has been demonstrated. The XPS studies show presence of $WO_3$ in the as-grown sample which indicates that the synthesis method for $WSe_2$ needs to be further refined to produce high quality flakes, the likes of which has been demonstrated for $WS_2$. In principle it has been shown that the method developed for $WS_2$ is transferable and can be further utilised in production in a number of other TMDC materials. 

The transfer of TMDC flakes as well as flakes of other 2D materials can prove to be a crucial element in a production route of any future device utilising these materials. We have therefore studied the effects of transferring as grown $WS_2$ flakes from one $SiO_2/Si$ substrate onto another $SiO_2/Si$ substrate to asses the quality of the transferred material and therefore the impact of the transfer. One of the most commonly used methods of wet transfer utilising PMMA has proven to alter the distribution of PL and Raman peaks across the samples. One of the most common effect cited in literature regarding the transfer is that of strain relaxation that can build up during CVD growth due to the high temperatures during the synthesis. Based on our experiments it cannot however be definitively concluded whether the transfer reduces or induces strain in the sample as a whole. There has been however demonstrated potential change in distribution of strain across the sample which manifests itself in shifts in PL peak positions as well as Raman peak positions. Any transfer attempts must therefore be considered carefully and preferably a different, more reliable transfer method be used.

In order to further study the PL behaviour in $WS_2$ flakes a low temperature studies were performed. It has been demonstrated that low pressure induces changes in PL behaviour due to desorption of $O_2$, $H_2O$ and $N_2$ molecules which otherwise result in p-doping. The exciton peak with FWHM of 32 meV has been observed which is higher than expected assuming the temperature is a main contributor to the width of the room temperature PL peak of 36 meV. A number of possible effects, such as sample heating due to poor thermal contacts or fogging of the chamber window can be attributed to the resulting width. Additionally a trion peak has been clearly resolved with binding energy to be about 45 meV.

By using same synthesis method vertical heterostructures of $Mo_2/WS_2$ have been grown. A single step, one pot growth has been achieved with a degree of tunability. The resulting heterostructure has been shown to be be more closely packed than a material produced more commonly via a mechanical transfer. This results in a greater degree of interaction between the layers which in turn manifests itself in both PL and Raman spectrscopy. The growth has then been also performed on a conducting substrate, such as gold. This allows such prepared heterostructure to be readily available as a photoelectrocatalyst for a reaction of water-splitting. Using developed understanding of Raman and PL spectroscopy it has been shown that at least 8$\%$ of the surface has been covered with heterostructures which leaves room for improvement and further refinement of the synthesis.

The developed synthesis methods as well as the characterisation understanding of the as grown samples can form a basis for further improvement of the performance of those materials. One of the promising approaches to enhancing the opto-electronic performance of devices is the use of plasmonic nanoparticles. These nanoparticles have been shown to enhance local electromagnetic field of incident light as well as reflect it to improve the absorption rate of light. This can therefore find application in solar cells or photoelectrocatalysis as presented in our work, the latter of which has not been presented yet.

Additionally further characterisation of the 1T'/2H systems is necessary to evaluate the stability. In particular synthesis of and systematic Raman mapping of large areas could prove useful in understanding these phases.