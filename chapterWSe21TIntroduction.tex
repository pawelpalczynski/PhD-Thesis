\chapter{1T' $WSe_2$ introduction}

The monolayers of TMDCs can be usually found in either 1H or 1T phase where the exact configuration depends on the thermodynamics \cite{Keum2015}\cite{Cho2015}. In order to identify which of those phases is more favourable the crystal field theory can be applied. The number of electrons in d orbitals of transition metals needs to be counted for a given TDMC material. For group 4 TMDCs (Ti, Zr and Hf) that number is 0 ($d^0$) while for group 10 (Ni, Pd and Pt) the number is 10 ($d^6$). It can be therefore shown that the number of electrons in d orbitals correlates with the phase transitions in TMDCs. The group 4 TMDCs ($d^0$) for instance are most stable in 1T phase. However the $d_{z^2}$ level of the 1H phase has the lowest energy and therefore the TMDCs of group 5 ($d_1$) and group 6 ($d_2$) fill that level first and therefore, they group 5 TMDCs are found most stable in  either 1T or 1H phase while the group 6 TMDCs are most favourable in 1H. After filling the $d_{z^2}$ level in group 6 TMDCs, the additional electron in group 7 TMDCs contribute to the 1T levels and results in a distorted octahedral coordination (1T') with clusterisation or Peierls distortions of the metal atoms. Furthermore the group 9 and 10 TMDCs are most stable in the regular octahedral coordination (1T).

Recently the 1T' phase has received attention due to the predictions of potential Weyl semimetal or quantum spin Hall effects for group 6 TMDCs \cite{Qian2014}\cite{Sun2015}. Additionally the lattice distortion orientation can be affected by strain which allows for mechanical control of ferroelasticity and potential application as a shape memory material \cite{Li2016}. In order to achieve those effects in group 6 TMDCs a phase transition from 1H to 1T' generally has to be occur \cite{Chhowalla2013}. Such transition is usually performed via the lithium intercalation or chemical exfoliation. Following the lithiation the excess of the electrons is found in the d orbitals of the transiton metals. As a result of that the electron density of those orbitals increases from $d_2$ to $d_{2+x}$. In order for a full transition from 1H to 1T to occur the excess of the electrons must be in range of 0.2 to 0.4. However if the excess is smaller than that then the distortion of the lattice may help to accommodate the phase transition.

One of the most important characteristics of TMDCs that influences a myriad of different properties is the number of layers. Especially in case of optical and electronic properties of 2H TMDCs the number of layers strongly affects the size and the nature of the bandgap, transitioning from indirect in bulk to direct in monolayer. The number of layers is however also of crucial importance to quantum effects observed or predicted in TMDCs in that the bulk material exhibits the in-plane inversion symmetry while the monolayers break it \cite{Saito2015}\cite{Lu2015}.